\documentclass[a4paper,12pt]{article}

\usepackage[T2A]{fontenc}
\usepackage[utf8]{inputenc}
\usepackage[english,russian]{babel}
\usepackage{amsmath,amsfonts,amssymb,amsthm,mathtools}
\usepackage{wasysym}
\author{Khaidukov Danila}
\title{Derivatives in \LaTeX{}}
\date{\today}

\begin{document}

\maketitle
\newpage

So, we're going to find the derivative of that interesting function:
\[ \frac{ \frac{ x + x ^ { 2 } \cdot 3 \cdot x - 3 }{ \sin\left( x\right)  } }{ \cos\left( 2 + x\right)  } \cdot \lg\left( x\right)  \]

But first of all lets try to simplify it!

LOL! No more easy

It's as easy as a breathe to understand that the derivative of VARIABLE is 1!

Elementary:
\[ \left(x\right)' = 1 \]

DECIMAL LOGARITHM is the best LOGARITHM because it's used for stellar magnetudes and decibels. So                                \[(\lg(f(x)))' = \frac{1}{f(x) \cdot ln(10)} \cdot f'(x) \]

So:
\[ \left(\lg\left( x\right) \right)' = \frac{ 1 }{ x \cdot \ln\left( 10\right)  } \cdot 1 \]

It's as easy as a breathe to understand that the derivative of VARIABLE is 1!

It's really easy to see:
\[ \left(x\right)' = 1 \]

Only a fool does not know that the derivative of NUMBER is 0!

Easy:
\[ \left(2\right)' = 0 \]

As we know the derivative of ADDITION is ADDITION of derivatives!

In that case:
\[ \left(2 + x\right)' = 0 + 1 \]

COSINUS is not very simple function, we'd use special formula                                \[(\cos(f(x))' = -\sin(f(x)) \cdot f'(x) \]

It's really easy to see:
\[ \left(\cos\left( 2 + x\right) \right)' = -1 \cdot \sin\left( 2 + x\right)  \cdot \left( 0 + 1\right)  \]

It's as easy as a breathe to understand that the derivative of VARIABLE is 1!

Then:
\[ \left(x\right)' = 1 \]

SINUS is really difficult function that's why we should use formula                                \[(\sin(f(x))' = \cos(f(x)) \cdot f'(x) \]

Then:
\[ \left(\sin\left( x\right) \right)' = \cos\left( x\right)  \cdot 1 \]

Only a fool does not know that the derivative of NUMBER is 0!

Clearly:
\[ \left(3\right)' = 0 \]

It's as easy as a breathe to understand that the derivative of VARIABLE is 1!

It's really easy to see:
\[ \left(x\right)' = 1 \]

Only a fool does not know that the derivative of NUMBER is 0!

You can see:
\[ \left(3\right)' = 0 \]

It's as easy as a breathe to understand that the derivative of VARIABLE is 1!

You can see:
\[ \left(x\right)' = 1 \]

Function to the POWER of number is interesting                                        \[\left(f^a(x)\right)' = a \cdot f^{a - 1}(x) \cdot f'(x)\]

Elementary:
\[ \left(x ^ { 2 }\right)' = 2 \cdot x ^ { 1 } \cdot 1 \]

It's clear that the derivative of MULTIPLICATION is given by formula                                \[(f(x) \cdot g(x))' = f'(x) \cdot g(x) + g'(x) \cdot f(x)\]

So:
\[ \left(x ^ { 2 } \cdot 3\right)' = 2 \cdot x ^ { 1 } \cdot 1 \cdot 3 + 0 \cdot x ^ { 2 } \]

It's clear that the derivative of MULTIPLICATION is given by formula                                \[(f(x) \cdot g(x))' = f'(x) \cdot g(x) + g'(x) \cdot f(x)\]

Elementary:
\[ \left(x ^ { 2 } \cdot 3 \cdot x\right)' = \left( 2 \cdot x ^ { 1 } \cdot 1 \cdot 3 + 0 \cdot x ^ { 2 }\right)  \cdot x + 1 \cdot x ^ { 2 } \cdot 3 \]

It's as easy as a breathe to understand that the derivative of VARIABLE is 1!

Elementary:
\[ \left(x\right)' = 1 \]

As we know the derivative of ADDITION is ADDITION of derivatives!

You can see:
\[ \left(x + x ^ { 2 } \cdot 3 \cdot x\right)' = 1 + \left( 2 \cdot x ^ { 1 } \cdot 1 \cdot 3 + 0 \cdot x ^ { 2 }\right)  \cdot x + 1 \cdot x ^ { 2 } \cdot 3 \]

Everybody knows that the derivative of SUBTRACT is SUBTRACT of derivatives!

Clearly:
\[ \left(x + x ^ { 2 } \cdot 3 \cdot x - 3\right)' = 1 + \left( 2 \cdot x ^ { 1 } \cdot 1 \cdot 3 + 0 \cdot x ^ { 2 }\right)  \cdot x + 1 \cdot x ^ { 2 } \cdot 3 - 0 \]

Not difficult to notice that the derivative of DIVISION is given by formula                                \[\left( \frac {f(x)}{g(x)} \right)' = \frac {f'(x) \cdot g(x) - g'(x) \cdot f(x)} {g ^ 2(x)}\]

So:
\[ \left(\frac{ x + x ^ { 2 } \cdot 3 \cdot x - 3 }{ \sin\left( x\right)  }\right)' = \frac{ \left( 1 + \left( 2 \cdot x ^ { 1 } \cdot 1 \cdot 3 + 0 \cdot x ^ { 2 }\right)  \cdot x + 1 \cdot x ^ { 2 } \cdot 3 - 0\right)  \cdot \sin\left( x\right)  - \cos\left( x\right)  \cdot 1 \cdot \left( x + x ^ { 2 } \cdot 3 \cdot x - 3\right)  }{ \sin ^ { 2 }\left( x\right)  } \]

Not difficult to notice that the derivative of DIVISION is given by formula                                \[\left( \frac {f(x)}{g(x)} \right)' = \frac {f'(x) \cdot g(x) - g'(x) \cdot f(x)} {g ^ 2(x)}\]

It's clear that:
\[ \left(\frac{ \frac{ x + x ^ { 2 } \cdot 3 \cdot x - 3 }{ \sin\left( x\right)  } }{ \cos\left( 2 + x\right)  }\right)' = \frac{ \frac{ \left( 1 + \left( 2 \cdot x ^ { 1 } \cdot 1 \cdot 3 + 0 \cdot x ^ { 2 }\right)  \cdot x + 1 \cdot x ^ { 2 } \cdot 3 - 0\right)  \cdot \sin\left( x\right)  - \cos\left( x\right)  \cdot 1 \cdot \left( x + x ^ { 2 } \cdot 3 \cdot x - 3\right)  }{ \sin ^ { 2 }\left( x\right)  } \cdot \cos\left( 2 + x\right)  - -1 \cdot \sin\left( 2 + x\right)  \cdot \left( 0 + 1\right)  \cdot \frac{ x + x ^ { 2 } \cdot 3 \cdot x - 3 }{ \sin\left( x\right)  } }{ \cos ^ { 2 }\left( 2 + x\right)  } \]

It's clear that the derivative of MULTIPLICATION is given by formula                                \[(f(x) \cdot g(x))' = f'(x) \cdot g(x) + g'(x) \cdot f(x)\]

Elementary:
\[ \left(\frac{ \frac{ x + x ^ { 2 } \cdot 3 \cdot x - 3 }{ \sin\left( x\right)  } }{ \cos\left( 2 + x\right)  } \cdot \lg\left( x\right) \right)' = \frac{ \frac{ \left( 1 + \left( 2 \cdot x ^ { 1 } \cdot 1 \cdot 3 + 0 \cdot x ^ { 2 }\right)  \cdot x + 1 \cdot x ^ { 2 } \cdot 3 - 0\right)  \cdot \sin\left( x\right)  - \cos\left( x\right)  \cdot 1 \cdot \left( x + x ^ { 2 } \cdot 3 \cdot x - 3\right)  }{ \sin ^ { 2 }\left( x\right)  } \cdot \cos\left( 2 + x\right)  - -1 \cdot \sin\left( 2 + x\right)  \cdot \left( 0 + 1\right)  \cdot \frac{ x + x ^ { 2 } \cdot 3 \cdot x - 3 }{ \sin\left( x\right)  } }{ \cos ^ { 2 }\left( 2 + x\right)  } \cdot \lg\left( x\right)  + \frac{ 1 }{ x \cdot \ln\left( 10\right)  } \cdot 1 \cdot \frac{ \frac{ x + x ^ { 2 } \cdot 3 \cdot x - 3 }{ \sin\left( x\right)  } }{ \cos\left( 2 + x\right)  } \]

Lets try to simplify our REALLY BIG derivative
\[ \frac{ \frac{ \left( 1 + \left( 2 \cdot x ^ { 1 } \cdot 1 \cdot 3 + 0 \cdot x ^ { 2 }\right)  \cdot x + 1 \cdot x ^ { 2 } \cdot 3 - 0\right)  \cdot \sin\left( x\right)  - \cos\left( x\right)  \cdot 1 \cdot \left( x + x ^ { 2 } \cdot 3 \cdot x - 3\right)  }{ \sin ^ { 2 }\left( x\right)  } \cdot \cos\left( 2 + x\right)  - -1 \cdot \sin\left( 2 + x\right)  \cdot \left( 0 + 1\right)  \cdot \frac{ x + x ^ { 2 } \cdot 3 \cdot x - 3 }{ \sin\left( x\right)  } }{ \cos ^ { 2 }\left( 2 + x\right)  } \cdot \lg\left( x\right)  + \frac{ 1 }{ x \cdot \ln\left( 10\right)  } \cdot 1 \cdot \frac{ \frac{ x + x ^ { 2 } \cdot 3 \cdot x - 3 }{ \sin\left( x\right)  } }{ \cos\left( 2 + x\right)  } \]

It's really easy to see:
\[ \frac{ \frac{ \left( 1 + \left( 2 \cdot x ^ { 1 } \cdot 1 \cdot 3 + 0 \cdot x ^ { 2 }\right)  \cdot x + 1 \cdot x ^ { 2 } \cdot 3\right)  \cdot \sin\left( x\right)  - \cos\left( x\right)  \cdot 1 \cdot \left( x + x ^ { 2 } \cdot 3 \cdot x - 3\right)  }{ \sin ^ { 2 }\left( x\right)  } \cdot \cos\left( 2 + x\right)  - -1 \cdot \sin\left( 2 + x\right)  \cdot 1 \cdot \frac{ x + x ^ { 2 } \cdot 3 \cdot x - 3 }{ \sin\left( x\right)  } }{ \cos ^ { 2 }\left( 2 + x\right)  } \cdot \lg\left( x\right)  + \frac{ 1 }{ x \cdot \ln\left( 10\right)  } \cdot 1 \cdot \frac{ \frac{ x + x ^ { 2 } \cdot 3 \cdot x - 3 }{ \sin\left( x\right)  } }{ \cos\left( 2 + x\right)  } \]

Clearly:
\[ \frac{ \frac{ \left( 1 + \left( 2 \cdot x ^ { 1 } \cdot 3 + 0 \cdot x ^ { 2 }\right)  \cdot x + 1 \cdot x ^ { 2 } \cdot 3\right)  \cdot \sin\left( x\right)  - \cos\left( x\right)  \cdot 1 \cdot \left( x + x ^ { 2 } \cdot 3 \cdot x - 3\right)  }{ \sin ^ { 2 }\left( x\right)  } \cdot \cos\left( 2 + x\right)  - -1 \cdot \sin\left( 2 + x\right)  \cdot 1 \cdot \frac{ x + x ^ { 2 } \cdot 3 \cdot x - 3 }{ \sin\left( x\right)  } }{ \cos ^ { 2 }\left( 2 + x\right)  } \cdot \lg\left( x\right)  + \frac{ 1 }{ x \cdot \ln\left( 10\right)  } \cdot 1 \cdot \frac{ \frac{ x + x ^ { 2 } \cdot 3 \cdot x - 3 }{ \sin\left( x\right)  } }{ \cos\left( 2 + x\right)  } \]

As you can see:
\[ \frac{ \frac{ \left( 1 + \left( 2x \cdot 3 + 0 \cdot x ^ { 2 }\right)  \cdot x + 1 \cdot x ^ { 2 } \cdot 3\right)  \cdot \sin\left( x\right)  - \cos\left( x\right)  \cdot 1 \cdot \left( x + x ^ { 2 } \cdot 3 \cdot x - 3\right)  }{ \sin ^ { 2 }\left( x\right)  } \cdot \cos\left( 2 + x\right)  - -1 \cdot \sin\left( 2 + x\right)  \cdot 1 \cdot \frac{ x + x ^ { 2 } \cdot 3 \cdot x - 3 }{ \sin\left( x\right)  } }{ \cos ^ { 2 }\left( 2 + x\right)  } \cdot \lg\left( x\right)  + \frac{ 1 }{ x \cdot \ln\left( 10\right)  } \cdot 1 \cdot \frac{ \frac{ x + x ^ { 2 } \cdot 3 \cdot x - 3 }{ \sin\left( x\right)  } }{ \cos\left( 2 + x\right)  } \]

Clearly:
\[ \frac{ \frac{ \left( 1 + \left( 2x \cdot 3 + 0\right)  \cdot x + 1 \cdot x ^ { 2 } \cdot 3\right)  \cdot \sin\left( x\right)  - \cos\left( x\right)  \cdot 1 \cdot \left( x + x ^ { 2 } \cdot 3 \cdot x - 3\right)  }{ \sin ^ { 2 }\left( x\right)  } \cdot \cos\left( 2 + x\right)  - -1 \cdot \sin\left( 2 + x\right)  \cdot 1 \cdot \frac{ x + x ^ { 2 } \cdot 3 \cdot x - 3 }{ \sin\left( x\right)  } }{ \cos ^ { 2 }\left( 2 + x\right)  } \cdot \lg\left( x\right)  + \frac{ 1 }{ x \cdot \ln\left( 10\right)  } \cdot 1 \cdot \frac{ \frac{ x + x ^ { 2 } \cdot 3 \cdot x - 3 }{ \sin\left( x\right)  } }{ \cos\left( 2 + x\right)  } \]

Clearly:
\[ \frac{ \frac{ \left( 1 + 2x \cdot 3 \cdot x + 1 \cdot x ^ { 2 } \cdot 3\right)  \cdot \sin\left( x\right)  - \cos\left( x\right)  \cdot 1 \cdot \left( x + x ^ { 2 } \cdot 3 \cdot x - 3\right)  }{ \sin ^ { 2 }\left( x\right)  } \cdot \cos\left( 2 + x\right)  - -1 \cdot \sin\left( 2 + x\right)  \cdot 1 \cdot \frac{ x + x ^ { 2 } \cdot 3 \cdot x - 3 }{ \sin\left( x\right)  } }{ \cos ^ { 2 }\left( 2 + x\right)  } \cdot \lg\left( x\right)  + \frac{ 1 }{ x \cdot \ln\left( 10\right)  } \cdot 1 \cdot \frac{ \frac{ x + x ^ { 2 } \cdot 3 \cdot x - 3 }{ \sin\left( x\right)  } }{ \cos\left( 2 + x\right)  } \]

Elementary:
\[ \frac{ \frac{ \left( 1 + 2x \cdot 3 \cdot x + x ^ { 2 } \cdot 3\right)  \cdot \sin\left( x\right)  - \cos\left( x\right)  \cdot 1 \cdot \left( x + x ^ { 2 } \cdot 3 \cdot x - 3\right)  }{ \sin ^ { 2 }\left( x\right)  } \cdot \cos\left( 2 + x\right)  - -1 \cdot \sin\left( 2 + x\right)  \cdot 1 \cdot \frac{ x + x ^ { 2 } \cdot 3 \cdot x - 3 }{ \sin\left( x\right)  } }{ \cos ^ { 2 }\left( 2 + x\right)  } \cdot \lg\left( x\right)  + \frac{ 1 }{ x \cdot \ln\left( 10\right)  } \cdot 1 \cdot \frac{ \frac{ x + x ^ { 2 } \cdot 3 \cdot x - 3 }{ \sin\left( x\right)  } }{ \cos\left( 2 + x\right)  } \]

It's clear that:
\[ \frac{ \frac{ \left( 1 + 2x \cdot 3 \cdot x + x ^ { 2 } \cdot 3\right)  \cdot \sin\left( x\right)  - \cos\left( x\right)  \cdot \left( x + x ^ { 2 } \cdot 3 \cdot x - 3\right)  }{ \sin ^ { 2 }\left( x\right)  } \cdot \cos\left( 2 + x\right)  - -1 \cdot \sin\left( 2 + x\right)  \cdot 1 \cdot \frac{ x + x ^ { 2 } \cdot 3 \cdot x - 3 }{ \sin\left( x\right)  } }{ \cos ^ { 2 }\left( 2 + x\right)  } \cdot \lg\left( x\right)  + \frac{ 1 }{ x \cdot \ln\left( 10\right)  } \cdot 1 \cdot \frac{ \frac{ x + x ^ { 2 } \cdot 3 \cdot x - 3 }{ \sin\left( x\right)  } }{ \cos\left( 2 + x\right)  } \]

Then:
\[ \frac{ \frac{ \left( 1 + 2x \cdot 3 \cdot x + x ^ { 2 } \cdot 3\right)  \cdot \sin\left( x\right)  - \cos\left( x\right)  \cdot \left( x + x ^ { 2 } \cdot 3 \cdot x - 3\right)  }{ \sin ^ { 2 }\left( x\right)  } \cdot \cos\left( 2 + x\right)  - -1 \cdot \sin\left( 2 + x\right)  \cdot \frac{ x + x ^ { 2 } \cdot 3 \cdot x - 3 }{ \sin\left( x\right)  } }{ \cos ^ { 2 }\left( 2 + x\right)  } \cdot \lg\left( x\right)  + \frac{ 1 }{ x \cdot \ln\left( 10\right)  } \cdot 1 \cdot \frac{ \frac{ x + x ^ { 2 } \cdot 3 \cdot x - 3 }{ \sin\left( x\right)  } }{ \cos\left( 2 + x\right)  } \]

You can see:
\[ \frac{ \frac{ \left( 1 + 2x \cdot 3 \cdot x + x ^ { 2 } \cdot 3\right)  \cdot \sin\left( x\right)  - \cos\left( x\right)  \cdot \left( x + x ^ { 2 } \cdot 3 \cdot x - 3\right)  }{ \sin ^ { 2 }\left( x\right)  } \cdot \cos\left( 2 + x\right)  - -1 \cdot \sin\left( 2 + x\right)  \cdot \frac{ x + x ^ { 2 } \cdot 3 \cdot x - 3 }{ \sin\left( x\right)  } }{ \cos ^ { 2 }\left( 2 + x\right)  } \cdot \lg\left( x\right)  + \frac{ 1 }{ x \cdot \ln\left( 10\right)  } \cdot \frac{ \frac{ x + x ^ { 2 } \cdot 3 \cdot x - 3 }{ \sin\left( x\right)  } }{ \cos\left( 2 + x\right)  } \]

EEEE! That looks like really easy nice
\[ \frac{ \frac{ \left( 1 + 2x \cdot 3 \cdot x + x ^ { 2 } \cdot 3\right)  \cdot \sin\left( x\right)  - \cos\left( x\right)  \cdot \left( x + x ^ { 2 } \cdot 3 \cdot x - 3\right)  }{ \sin ^ { 2 }\left( x\right)  } \cdot \cos\left( 2 + x\right)  - -1 \cdot \sin\left( 2 + x\right)  \cdot \frac{ x + x ^ { 2 } \cdot 3 \cdot x - 3 }{ \sin\left( x\right)  } }{ \cos ^ { 2 }\left( 2 + x\right)  } \cdot \lg\left( x\right)  + \frac{ 1 }{ x \cdot \ln\left( 10\right)  } \cdot \frac{ \frac{ x + x ^ { 2 } \cdot 3 \cdot x - 3 }{ \sin\left( x\right)  } }{ \cos\left( 2 + x\right)  } \]

Derivatives are really easy!

\end{document}
